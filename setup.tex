%%%%%%%%%%%%%%%%%%%%%%%%%%%%%%%%%%%%%%%%%%%%%%%%%%%%%%%%%%%%%%%%%%%%
%% I, the copyright holder of this work, release this work into the
%% public domain. This applies worldwide. In some countries this may
%% not be legally possible; if so: I grant anyone the right to use
%% this work for any purpose, without any conditions, unless such
%% conditions are required by law.
%%%%%%%%%%%%%%%%%%%%%%%%%%%%%%%%%%%%%%%%%%%%%%%%%%%%%%%%%%%%%%%%%%%%

\documentclass[
  digital, %% This option enables the default options for the
           %% digital version of a document. Replace with `printed`
           %% to enable the default options for the printed version
           %% of a document.
  table,   %% Causes the coloring of tables. Replace with `notable`
           %% to restore plain tables.
  nolof,     %% Prints the List of Figures. Replace with `nolof` to
           %% hide the List of Figures.
  nolot,     %% Prints the List of Tables. Replace with `nolot` to
           %% hide the List of Tables.
  %% More options are listed in the user guide at
  %% <http://mirrors.ctan.org/macros/latex/contrib/fithesis/guide/mu/fi.pdf>.
]{fithesis3}
%% The following section sets up the locales used in the thesis.
\usepackage[resetfonts]{cmap} %% We need to load the T2A font encoding
\usepackage[T1,T2A]{fontenc}  %% to use the Cyrillic fonts with Russian texts.
\usepackage[slovak]{babel}        %% foreign texts to be typeset as follows:

%%
%% The following section sets up the metadata of the thesis.
\thesissetup{
    date          = \the\year/\the\month/\the\day,
    university    = mu,
    faculty       = fi,
    type          = bc,
    author        = Martin Styk,
    gender        = m,
    advisor       = {Ing. Mgr. et Mgr. Zdeněk Říha, Ph.D.},
    title         = {Analýza inštalačných APK súborov pre OS Android},
    keywords      = {APK súbor, Android, Apktool, malvér, analýza aplikácií, AndroidManifest.xml},
}
\thesislong{abstract}{
    Práca sa zaoberá získavaním metadát o~inštalačných APK súboroch pre mobilný operačný systém Android. V~rámci práce je vytvorená rozsiahla databáza APK balíčkov. Na základe analýzy týchto súborov sú určené štatistické vlastnosti APK súborov a príslušných aplikácií. Ako súčasť tejto práce sú implementované nástroje na hromadné sťahovanie APK súborov, ich analýzu a výpočet štatistických dát nad množinou APK súborov. Práca sa zaoberá aj bezpečnosťou aplikácií a detekciou modifikovaných APK súborov. V~práci je navrhnutá metóda detekcie upravených a prebalených APK balíčkov, ktorá je aj prakticky implementovaná. V~teoretickej časti je popísaná štruktúra APK balíčkov.
}
\thesislong{thanks}{
Rád by som sa poďakoval vedúcemu práce Ing. Mgr. et Mgr. Zdeňkovi Říhovi, Ph.D. za venovaný čas, ochotu a cenné pripomienky, ktoré mi pomohli pri tvorbe tejto práce.
}
%%XML
\usepackage{listings}

\usepackage{color}
\definecolor{gray}{rgb}{0.4,0.4,0.4}
\definecolor{light-gray}{rgb}{0,0,0}
\definecolor{darkblue}{rgb}{0.0,0.0,0.6}
\definecolor{cyan}{rgb}{0.0,0.6,0.6}


\lstset{
  basicstyle=\ttfamily,
  columns=fullflexible,
  showstringspaces=false,
  captionpos=b,
}

\lstdefinelanguage{XML}
{
  morestring=[b]",
  morestring=[s]{>}{<},
  morecomment=[s]{<?}{?>},
  stringstyle=\color{black},
  identifierstyle=\color{darkblue},
  keywordstyle=\color{cyan},
  morekeywords={xmlns,version,type}% list your attributes here
}
\renewcommand{\lstlistingname}{Kód}

%% The following section sets up the bibliography.
\usepackage{csquotes}
\usepackage[              %% When typesetting the bibliography, the
  backend=biber,          %% `numeric` style will be used for the
  style=iso-numeric,          %% entries and the `numeric-comp` style
  citestyle=numeric-comp, %% for the references to the entries. The
  sorting=none,           %% entries will be sorted in cite order.
  sortlocale=auto,        %% For more unformation about the available
]{biblatex}               %% `style`s and `citestyles`, see:
%% <http://mirrors.ctan.org/macros/latex/contrib/biblatex/doc/biblatex.pdf>.
\addbibresource{bibliografie.bib} %% The bibliograpic database within
                          %% the file `example.bib` will be used.
\usepackage{makeidx}      %% The `makeidx` package contains
\makeindex                %% helper commands for index typesetting.
%% These additional packages are used within the document:
\usepackage{paralist}
\usepackage{amsmath}
\usepackage{amsthm}
\usepackage{amsfonts}
\usepackage{url}
\usepackage{menukeys}
\usepackage{fancybox}
\usepackage{pgf-pie}
\usepackage{tikz}
\usepackage{longtable}
\usepackage{tabularx}

\newcommand{\zv}{\textit}
\newcommand{\cesta}{\zv}
\newcommand{\bod}{\zv}