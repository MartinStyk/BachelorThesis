%%%%%%%%%%%%%%%%%%%%%%%%%%%%%%%%%%%%%%%%%%%%%%%%%%%%%%%%%%%%%%%%%%%%
%% I, the copyright holder of this work, release this work into the
%% public domain. This applies worldwide. In some countries this may
%% not be legally possible; if so: I grant anyone the right to use
%% this work for any purpose, without any conditions, unless such
%% conditions are required by law.
%%%%%%%%%%%%%%%%%%%%%%%%%%%%%%%%%%%%%%%%%%%%%%%%%%%%%%%%%%%%%%%%%%%%

\documentclass[
  digital, %% This option enables the default options for the
           %% digital version of a document. Replace with `printed`
           %% to enable the default options for the printed version
           %% of a document.
  table,   %% Causes the coloring of tables. Replace with `notable`
           %% to restore plain tables.
  lof,     %% Prints the List of Figures. Replace with `nolof` to
           %% hide the List of Figures.
  lot,     %% Prints the List of Tables. Replace with `nolot` to
           %% hide the List of Tables.
  %% More options are listed in the user guide at
  %% <http://mirrors.ctan.org/macros/latex/contrib/fithesis/guide/mu/fi.pdf>.
]{fithesis3}
%% The following section sets up the locales used in the thesis.
\usepackage[resetfonts]{cmap} %% We need to load the T2A font encoding
\usepackage[T1,T2A]{fontenc}  %% to use the Cyrillic fonts with Russian texts.
\usepackage[slovak]{babel}        %% foreign texts to be typeset as follows:

%%
%% The following section sets up the metadata of the thesis.
\thesissetup{
    date          = \the\year/\the\month/\the\day,
    university    = mu,
    faculty       = fi,
    type          = bc,
    author        = Martin Styk,
    gender        = m,
    advisor       = {Ing. Mgr. et Mgr. Zdeněk Říha, Ph.D.},
    title         = {Analýza inštalačných APK súborov pre OS Android},
    keywords      = {APK súbor, Android, Apktool, malvér, analýza aplikácií, AndroidManifest.xml},
}
\thesislong{abstract}{
    Práca sa zaoberá získavaním metadát o inštalačných APK súboroch pre mobilný operačný systém Android. V rámci práce je vytvorená rozsiahla databáza APK balíčkov. Na základe analýzy týchto súborov sú určené štatistické vlastnosti APK súborov a príslušných aplikácií. Ako súčasť tejto práce je implementovaný nástroj na hromadné sťahovanie APK súborov, ich analýzu a výpočet štatistických dát nad množinou APK súborov. Práca sa zaoberá aj bezpečnosťou aplikácií a detekciou modifikovaných APK súborov. V práci je navrhnutá metóda detekcie upravených a prebalených APK balíčkov, ktorá je aj prakticky implementovaná. 
    V teoretickej časti je popísaná štruktúra APK balíčkov a súborov v nich obsiahnutych. 
}
\thesislong{thanks}{
    Rád by som sa poďakoval vedúcemu práce Ing. Mgr. et Mgr. Zdeňkovi Říhovi, Ph.D. za venovaný čas, ochotu a cenné pripomienky, ktoré mi pomohli pri tvorbe tejto práce.
}
%% The following section sets up the bibliography.
\usepackage{csquotes}
\usepackage[              %% When typesetting the bibliography, the
  backend=biber,          %% `numeric` style will be used for the
  style=numeric,          %% entries and the `numeric-comp` style
  citestyle=numeric-comp, %% for the references to the entries. The
  sorting=none,           %% entries will be sorted in cite order.
  sortlocale=auto         %% For more unformation about the available
]{biblatex}               %% `style`s and `citestyles`, see:
%% <http://mirrors.ctan.org/macros/latex/contrib/biblatex/doc/biblatex.pdf>.
\addbibresource{example.bib} %% The bibliograpic database within
                          %% the file `example.bib` will be used.
\usepackage{makeidx}      %% The `makeidx` package contains
\makeindex                %% helper commands for index typesetting.
%% These additional packages are used within the document:
\usepackage{paralist}
\usepackage{amsmath}
\usepackage{amsthm}
\usepackage{amsfonts}
\usepackage{url}
\usepackage{menukeys}
\usepackage{fancybox}

\begin{document}
\chapter{Úvod}
TBD

%% Operačný systém Android 
\chapter{Operačný systém Android}
Android je mobilný operačný systém navrhnutý primárne pre zariadenia s~dotykovou obrazovkou. Android je dominantným operačným systémom na trhu s~mobilnými zariadeniami ako sú chytré telefóny a tablety. V~treťom kvartáli roku 2015 dosahoval až 84,7\% podiel na trhu operačných systémov pre mobilné zariadenia~\cite{Westenberg2015}. Systém je založený na linuxovom jadre.

Android je aktuálne vyvíjaný spoločnosťou Google ako open source projekt. Existuje aktívna komunita vývojárov podieľajúca sa na vývoji projektu Android Open Source Project. Väčšina Android zariadení sa predáva s~kombináciou open source a proprietárneho softvéru. Medzi proprietárne časti zdrojového kódu patria nadstavby výrobcov telefónov a služby spoločnosti Google, tzv. Google services. Android nemá žiadny centralizovaný systém aktualizácií. To má za následok, že veľká časť zariadení nedostáva aktualizácie dostatočne často. Až 87,7\,\% zariadení obsahuje kritické bezpečnostné zraniteľnosti, ktoré sú známe, ale nie sú opravené kvôli slabej podpore~\cite{Thomas2015}.


\section{História}
Začiatok vývoja operačného systému, ktorý je dnes známy ako Android siaha do roku 2003, kedy vznikla spoločnosť Android, Inc. Prvotným zámerom spoločnosti bola tvorba systému pre digitálne fotoaparáty, avšak kvôli malému trhu bol vývoj preorientovaný na operačný systém pre mobilné zariadenia. Zakladatelia spoločnosti Andy Rubin, Rich Miner, Nick Sears  a Chris White plánovali vývoj operačného systému pre chytré mobilné zariadenia, ktoré dokážu efektívne využívať prostredie a preferencie užívateľov~\cite{Beavis2008}. Spoločnosť Android bola v~roku 2005 kúpená spoločnosťou Google za približne 50 miliónov dolárov~\cite{Rosoff2011}. 5.\,novembra 2007 bolo predstavené konzorcium \zv{Open Handset Alliance}, skladajúce sa z~výrobcov mobilných zariadení, mobilných operátorov a výrobcov komponentov pre mobilné zariadenia. Hlavným cieľom konzorcia je vývoj otvorených mobilných štandardov~\cite{OHA}. Prvým predstaveným produktom tejto skupiny bol operačný systém založený na linuxovom jadre -- Android. Prvým komerčne dostupným chytrým telefónom s~operačným systémom Android sa 22.\,novembra 2008 stal \zv{HTC Dream}. Od roku 2008 sa systém inkrementálne vylepšuje a vyvíja. Bolo vydaných množstvo opráv, vylepšení a nových funkcií. Začínajúc od verzie \zv{Android 1.5 Cupcake}, je každá verzia pomenovaná podľa cukroviniek.

\section{Architektúra systému}
Operačný systém Android je možné dekomponovať do piatich sekcií a štyroch základných architektonických vrstiev organizovaných v~zásobníkovej štruktúre~\cite{Gunasekera2012} zobrazenej na obrázku \ref{fig:struktura}.
\begin{figure} [htb]
 \centering
	\ovalbox{
		\begin{minipage}[b]{10cm}	
			\begin{center}
			Aplikácie
			\end{center}					
		\end{minipage}
		}\\
	\ovalbox{
		\begin{minipage}[b]{10cm}	
			\begin{center}
			Aplikačný rámec
			\end{center}					
		\end{minipage}
		}\\
	\ovalbox{
		\begin{minipage}[b]{10cm}	
			\begin{center}
			Knižnice \hfill Android Runtime (DVM)
			\end{center}
		\end{minipage}
		}\\
	\ovalbox{
		\begin{minipage}[b]{10cm}	
			\begin{center}
			Linuxové jadro
			\end{center}					
		\end{minipage}
		}	
  \caption{Vrstevnatá architektúra systému Android}
  \label{fig:struktura}
\end{figure}
\subsection{Linuxové jadro}
Najnižšiu vrstvu predstavuje Linux jadro vo verzií 2.6. Jadro je upravené za účelom optimalizácie spotreby energie a operačnej pamäte, podporuje preemptívny multitasking. Táto vrstva poskytuje abstrakciu medzi hardvérom zariadenia a vyššími softvérovými vrstvami~\cite{Allen2010}. Na tejto vrstve sa nachádzajú ovládače hardvérových komponent ako fotoaparát, dotyková obrazovka alebo sieťové rozhranie~\cite{architecture}.
\subsection{Android Runtime a Dalvik Virtual Machine}
\zv{Dalvik Virtual Machine} je virtuálny stroj slúžiaci na exekúciu Android aplikácií~\cite{dalvik}. Je obdobou virtuálneho stroja \zv{JVM (Java Virtual Machine)} používaného pri jazyku Java. Virtuálny stroj \zv{Dalvik} využíva nízkoúrovňovú funkcionalitu linuxového jadra. Každá aplikácia je spustená vo vlastnom procese a na vlastnej inštancii virtuálneho stroja. Tento prístup zaručuje, že aplikácie sa navzájom neúmyselne neovplyvňujú, nepristupujú priamo k~hardvéru zariadenia a využívajú abstrakciu, ktorá zabezpečuje ich platformovú nezávislosť~\cite{architecture}.  Od verzie Android 5.0 je virtuálny stroj \zv{Dalvik} plne nahradený novým behovým prostredím \zv{Android Runtime} (\zv{ART}).
\subsection{Knižnice}
Android obsahuje množstvo knižníc využívaných vývojármi alebo samotným systémom. Špecifickou skupinou sú natívne knižnice jadra, často označované ako \zv{Dalvik knižnice}, ktoré obsahujú kód pre interakciu s~inštanciou virtuálneho stroja, ale aj napríklad knižnice pre prístup k~systému súborov. Veľká časť knižníc obsiahnutá v~tejto vrstve využíva natívny kód v~jazyku C/C++. Takéto knižnice slúžia ako obal okolo C/C++ kódu, ku ktorému  sprostredkúvajú prístup pomocou jazyka Java. Táto vrstva obsahuje niektoré štandardné knižnice známe z~jazyka Java, ktoré sú upravené pre využitie na operačnom systéme Android, ale aj knižnice špecifické pre platformu Android, tzv. \zv{Android knižnice}~\cite{Hashimi2009}.
\subsection{Aplikačný rámec}
Vrstva aplikačného rámca poskytuje vysokoúrovňové služby používané na manažment aplikácie. Využíva koncept Android aplikácií, ktoré sa skladajú z~viacerých komponent. Kľúčové služby poskytované aplikačným rámcom sú~\cite{architecture}: 
\begin{itemize}
	\item \bod{Activity Manager}  --  ovláda životný cyklus aktivít a spravuje zásobník naposledy spustených aktivít,
	\item \bod{Content Provider}  --  umožňuje zdieľanie dát medzi aplikáciami,
	\item \bod{Resource Manager}  --  poskytuje prístup k~zdrojovým súborom ako reťazce, obrázky, dizajny obrazoviek,
	\item \bod{Notification Manager}  --  umožňuje aplikácii zobrazovať upozornenia,
	\item \bod{View system}  --  poskytuje prvky grafického používateľského rozhrania aplikácie,
	\item \bod{Package Manager}  --  umožňuje aplikáciám získať informácie o~ostatných aplikáciách nainštalovaných na zariadení,
	\item \bod{Telephony Manager}  --  umožňuje aplikáciám získať informácie o~stave telefónnych služieb,
	\item \bod{Location Manager}  --  poskytuje aplikáciám informácie o~polohe zariadenia.
\end{itemize}
\subsection{Vrstva aplikácií}
Na vrchole vrstevnatej architektúry systému Android sú aplikácie, ktoré využívajú súčinnosť všetkých spomenutých vrstiev. 

\section{Aplikácie}
\subsection{Distribúcia aplikácií}

APK súbory predstavujúce inštalačné balíčky Android aplikácií sú najčastejšie distribuované pomocou obchodu s~aplikáciami. Oficiálny obchod pre Android zariadenia je \zv{Google Play}. Aplikácie môžu byť distribuované pomocou alternatívnych obchodov ako napríklad \zv{Amazon Appstore} alebo \zv{SlideMe}. Operačný systém Android v~základnom nastavení neumožňuje inštaláciu aplikácií z~iných zdrojov ako \zv{Google Play}. Inštalácia z~neznámych zdrojov môže byť povolená v~nastaveniach zariadenia. Inštalačné APK balíčky je možné získať aj zo stránok na zdieľanie ľubovoľného obsahu, na ktorých sa často distribuujú aplikácie ktoré sú v~oficiálnych zdrojoch platené. Takéto súbory sú často modifikované a môžu obsahovať potenciálny škodlivý kód. Často ich označujeme ako prebalené aplikácie. Viac informácií o~prebalených aplikáciách sa nachádza v~kapitole \ref{Repackaged}.

\subsection{Inštalácia aplikácií}
Aplikácie môžeme rozdeliť na dve skupiny:
\begin{itemize}
\item Predinštalované aplikácie – často označované aj ako systémové aplikácie. Tieto aplikácie sú nainštalované spolu so systémom a často nemôžu byť bežným používateľom odinštalované. Príkladom je základná aplikácia pre fotoaparát, kontakty alebo telefón.
\item Aplikácie nainštalované používateľom – Aplikácie nainštalované jedným z~nasledujúcich spôsobov~\cite{Elenkov2015}:
\begin{itemize}
\item prostredníctvom obchodu s~aplikáciami, najčastejšie \zv{Google Play Store},
\item prostredníctvom nástroja \zv{Android Debug Bridge} (\zv{ADB}) ktorý je obsiahnutý, v~\zv{Android SDK} a umožňuje inštaláciu a ladenie aplikácií na zariadení pripojenom k~počítaču pomocou USB kábla
\item otvorením APK balíčka umiestneného v~zariadení.
\end{itemize}
\end{itemize}

\noindent Základnou aplikáciou starajúcou sa o~inštaláciu APK balíčkov je \zv{PackageInstaller}, ktorý poskytuje užívateľské rozhranie na komunikáciu so službou \zv{PackageManager}. \zv{PackageManager}  poskytuje v~rámci triedy \zv{PackageManagerService.java} API pre inštaláciu, aktualizáciu a odinštaláciu aplikácií. Natívny démon \zv{installd} prijíma požiadavky od služby \zv{PackageManagerService.java} s~ktorou komunikuje prostredníctvom lokálneho soketu \cesta{/dev/socket/installed}. Služba \zv{PackageManager} a démon \zv{installd} sú spustené pri štarte systému. \zv{PackageManager} čaká na pridanie požiadavky na inštaláciu do zoznamu inštalovaných aplikácií. Pri inštalácii analyzuje súbor \zv{AndroidManifest.xml} a relevantné informácie ukladá do súborov \cesta{/data/system/packages.xml} a \cesta{/data/system/packages.list}. Priečinok, do ktorého sa APK súbor rozbalí, vytvára démon \zv{installd}, o~rozbalenie a kopírovanie obsahu sa stará \zv{PackageManager}. Predinštalované (systémové) aplikácie sú inštalované do zložky \cesta{/system/app/}, aplikácie inštalované užívateľom do zložky \cesta{/data/app/}. Súbor \zv{classes.dex}, ktorý je obsiahnutý v~APK balíčku je kopírovaný do \cesta{/data/dalvik-cache/}. \zv{PackageManager} vytvorí priečinok \cesta{/data/data/nazov\_balíčku} v~ktorom sa nachádzajú preferencie, databázy alebo natívne knižnice aplikácie~\cite{Parmar2013}.



\chapter{These are}
\section{the available}
\subsection{sectioning commands.}
\paragraph{Paragraphs and}
\subparagraph{subparagraphs are available as well.}
Inside the text, you can also use unnumbered lists,
\begin{itemize}
  \item such as
  \item this one
  \begin{itemize}
    \item     and they can be nested as well.
    \item[>>] You can even turn the bullets into something fancier,
    \item[\S] if you so desire.
  \end{itemize}
\end{itemize}
Numbered lists are
\begin{enumerate}
  \item very
  \begin{enumerate}
    \item similar
  \end{enumerate}
\end{enumerate}
and so are description lists:
\begin{description}
  \item[Description list]
    A list of terms with a description of each term
\end{description}
The spacing of these lists is geared towards paragraphs of text.
For lists of words and phrases, the \textsf{paralist} package
offers commands
\begin{compactitem}
  \item that
  \begin{compactitem}
    \item are
    \begin{compactitem}
      \item better
      \begin{compactitem}
        \item suited
      \end{compactitem}
    \end{compactitem}
  \end{compactitem}
\end{compactitem}
\begin{compactenum}
  \item to
  \begin{compactenum}
    \item this
    \begin{compactenum}
      \item kind of
      \begin{compactenum}
        \item content.
      \end{compactenum}
    \end{compactenum}
  \end{compactenum}
\end{compactenum}


\chapter{Floats and references}
\begin{figure}
  \begin{center}
    %% PNG and JPG images can be inserted into the document as well,
    %% but their resolution needs to be adequate. The minimum is
    %% about 250 pixels per 1 centimeter. That means that a JPG or
    %% PNG image typeset at 40 × 40 mm should be 1000 × 1000 px
    %% large at minimum.
    \includegraphics[width=40mm]{fithesis/logo/mu/fithesis-base.pdf}
  \end{center}
  \caption{The logo of the Masaryk University at 40\,mm}
  \label{fig:mulogo1}
\end{figure}

\begin{figure}
  \begin{minipage}{.66\textwidth}
    \includegraphics[width=\textwidth]{fithesis/logo/mu/fithesis-base.pdf}
  \end{minipage}
  \begin{minipage}{.33\textwidth}
    \includegraphics[width=\textwidth]{fithesis/logo/mu/fithesis-base.pdf} \\
    \includegraphics[width=\textwidth]{fithesis/logo/mu/fithesis-base.pdf}
  \end{minipage}
  \caption{The logo of the Masaryk University at $\frac23$ and
    $\frac13$ of text width}
  \label{fig:mulogo2}
\end{figure}

\begin{table}
  \begin{tabularx}{\textwidth}{lllX}
    \toprule
    Day & Min Temp & Max Temp & Summary \\
    \midrule
    Monday & $13^{\circ}\mathrm{C}$ & $21^\circ\mathrm{C}$ & A
    clear day with low wind and no adverse current advisories. \\
    Tuesday & $11^{\circ}\mathrm{C}$ & $17^\circ\mathrm{C}$ & A
    trough of low pressure will come from the northwest. \\
    Wednesday & $10^{\circ}\mathrm{C}$ &
    $21^\circ\mathrm{C}$ & Rain will spread to all parts during the
    morning. \\
    \bottomrule
  \end{tabularx}
  \caption{A weather forecast}
  \label{tab:weather}
\end{table}

The logo of the Masaryk University is shown in Figure
\ref{fig:mulogo1} and Figure \ref{fig:mulogo2} at pages
\pageref{fig:mulogo1} and \pageref{fig:mulogo2}. The weather
forecast is shown in Table \ref{tab:weather} at page
\pageref{tab:weather}. The following chapter is Chapter
\ref{chap:matheq} and starts at page \pageref{chap:matheq}.
Items \ref{item:star1}, \ref{item:star2}, and
\ref{item:star3} are starred in the following list:
\begin{compactenum}
  \item some text
  \item some other text
  \item $\star$ \label{item:star1}
  \begin{compactenum}
    \item some text
    \item $\star$ \label{item:star2}
    \item some other text
    \begin{compactenum}
      \item some text
      \item some other text
      \item yet another piece of text
      \item $\star$ \label{item:star3}
    \end{compactenum}
    \item yet another piece of text
  \end{compactenum}
  \item yet another piece of text
\end{compactenum}
If your reference points to a place that has not yet been typeset,
the \verb"\ref" command will expand to \textbf{??} during the first
run of
\texttt{pdflatex \jobname.tex}
and a second run is going to be needed for the references to
resolve. With online services -- such as Overleaf -- this is
performed automatically.

\chapter{Mathematical equations}
\label{chap:matheq}
\TeX{} comes pre-packed with the ability to typeset inline
equations, such as $\mathrm{e}^{ix}=\cos x+i\sin x$, and display
equations, such as \[
  \mathbf{A}^{-1} = \begin{bmatrix}
  a & b \\ c & d \\
  \end{bmatrix}^{-1} =
  \frac{1}{\det(\mathbf{A})} \begin{bmatrix}
  \,\,\,d & \!\!-b \\ -c & \,a \\
  \end{bmatrix} =
  \frac{1}{ad - bc} \begin{bmatrix}
  \,\,\,d & \!\!-b \\ -c & \,a \\
  \end{bmatrix}.
\] \LaTeX{} defines the automatically numbered \texttt{equation}
environment:
\begin{equation}
  \gamma Px = PAx = PAP^{-1}Px.
\end{equation}
The package \textsf{amsmath} provides several additional
environments that can be used to typeset complex equations:
\begin{enumerate}
  \item An equation can be spread over multiple lines using the
    \texttt{multline} environment:
    \begin{multline}
      a + b + c + d + e + f + b + c + d + e + f + b + c + d + e +
f \\
      + f + g + h + i + j + k + l + m + n + o + p + q
    \end{multline}

  \item Several aligned equations can be typeset using the
    \texttt{align} environment:
    \begin{align}
              a + b &= c + d     \\
                  u &= v + w + x \\[1ex]
      i + j + k + l &= m
    \end{align}

  \item The \texttt{alignat} environment is similar to
    \texttt{align}, but it doesn't insert horizontal spaces between
    the individual columns:
    \begin{alignat}{2}
      a + b + c &+ d       &   &= 0 \\
              e &+ f + g   &   &= 5
    \end{alignat}

  \item Much like chapter, sections, tables, figures, or list
    items, equations -- such as \eqref{eq:first} and
    \eqref{eq:mine} -- can also be labeled and referenced:
    \begin{alignat}{4}
      b_{11}x_1 &+ b_{12}x_2  &  &+ b_{13}x_3  &  &             &
        &= y_1,                   \label{eq:first} \\
      b_{21}x_1 &+ b_{22}x_2  &  &             &  &+ b_{24}x_4  &
        &= y_2. \tag{My equation} \label{eq:mine}
    \end{alignat}

  \item The \texttt{gather} environment makes it possible to
    typeset several equations without any alignment:
    \begin{gather}
      \psi = \psi\psi, \\
      \eta = \eta\eta\eta\eta\eta\eta, \\
      \theta = \theta.
    \end{gather}

  \item Several cases can be typeset using the \texttt{cases}
    environment:
    \begin{equation}
      |y| = \begin{cases}
        \phantom-y & \text{if }z\geq0, \\
                -y & \text{otherwise}.
      \end{cases}
    \end{equation}
\end{enumerate}
For the complete list of environments and commands, consult the
\textsf{amsmath} package manual\footnote{
  See \url{http://mirrors.ctan.org/macros/latex/required/amslatex/math/amsldoc.pdf}.
  The \texttt{\textbackslash url} command is provided by the
  package \textsf{url}.
}.

\chapter{\textnormal{We \textsf{have} \texttt{several} \textsc{fonts}
  \textit{at} \textbf{disposal}}}
The serified roman font is used for the main body of the text.
\textit{Italics are typically used to denote emphasis or
quotations.} \texttt{The teletype font is typically used for source
code listings.} The \textbf{bold}, \textsc{small-caps} and
\textsf{sans-serif} variants of the base roman font can be used to
denote specific types of information.

\tiny We \scriptsize can \footnotesize also \small change \normalsize
the \large font \Large size, \LARGE although \huge it \Huge
is \huge usually \LARGE not \Large necessary.\normalsize

A wide variety of mathematical fonts is also available, such as: \[
  \mathrm{ABC}, \mathcal{ABC}, \mathbf{ABC}, \mathsf{ABC},
  \mathit{ABC}, \mathtt{ABC}
\] By loading the \textsf{amsfonts} packages, several additional
fonts will become available: \[
  \mathfrak{ABC}, \mathbb{ABC}
\] Many other mathematical fonts are available\footnote{
  See \url{http://tex.stackexchange.com/a/58124/70941}.
}.

\chapter{Inserting the bibliography}
After loading the \texttt{biblatex} package and linking a
bibliography data\-base file to the document using the
\verb"\addbibresource" command, you can start citing the entries.
This is just dummy text \cite{inbook-full} lightly sprinkled with
citations \cite[p.~123]{incollection-full}.  Several sources can be
cited at once \cite{whole-collection, manual-minimal,manual-full}.
\citetitle{inbook-full} was written by \citeauthor{inbook-full} in
\citeyear{inbook-full}. We can also produce \textcite{inbook-full}
or%% Let us define a compound command:
\def\citeauthoryear#1{(\textcite{#1},~\citeyear{#1})}
\citeauthoryear{inbook-full}. The full bibliographic citation is:
\emph{\fullcite{inbook-full}}. We can easily insert a bibliographic
citation into the footnote\footfullcite{inbook-full}.

The \verb"\nocite" command will not generate any
output\nocite{booklet-full}, but it will insert its argument into
the bibliography. The \verb"\nocite{*}" command will insert all the
records in the bibliography database file into the bibliography.
Try uncommenting the command
%% \nocite{*}
and watch the bibliography section come apart at the seams.

When typesetting the document for the first time, citing a
\texttt{work} will expand to [\textbf{work}] and the
\verb"\printbibliography" command will produce no output. It is now
necessary to generate the bibliography by running \texttt{biber
\jobname.bcf} from the command line and then by typesetting the
document again twice. During the first run, the bibliography
section and the citations will be typeset, and in the second run,
the bibliography section will appear in the table of contents.

The \texttt{biber} command needs to be executed from within the
directory, where the \LaTeX\ source file is located. In Windows,
the command line can be opened in a directory by holding down the
\keys{Shift} key and by clicking the right mouse button while
hovering the cursor over a directory.  Select the \menu{Open
Command Window Here} option in the context menu that opens shortly
afterwards.

With online services -- such as Overleaf -- all commands are
executed automatically.

{\csname captions\languagename\endcsname %% Temporarily override
%% the BibLaTeX localization with the original babel definitions.
\makeatletter %% Use the correct localization of the quotations.
  \thesis@selectLocale{\thesis@locale}\makeatother
\printbibliography[heading=bibintoc]} %% Print the bibliography.

\chapter{Inserting the index}
After using the \verb"\makeindex" macro and loading the
\texttt{makeidx} package that provides additional indexing
commands, index entries can be created by issuing the \verb"\index"
command. \index{dummy text|(}It is possible to create ranged index
entries, which will encompass a span of text.\index{dummy text|)}
To insert complex typographic material -- such as $\alpha$
\index{alpha@$\alpha$} or \TeX{} \index{TeX@\TeX} --
into the index, you need to specify a text string, which will
determine how the entry will be sorted. It is also possible to
create hierarchal entries. \index{vehicles!trucks}
\index{vehicles!speed cars}

After typesetting the document, it is necessary to generate the
index by running
\begin{center}%
  \texttt{texindy -I latex -C utf8 -L }$\langle$\textit{locale}%
  $\rangle$\texttt{ \jobname.idx}
\end{center}
from the command line, where $\langle$\textit{locale}$\rangle$
corresponds to the main locale of your thesis -- such as
\texttt{english}, and then typesetting the document again.

The \texttt{texindy} command needs to be executed from within the
directory, where the \LaTeX\ source file is located. In Windows,
the command line can be opened in a directory by holding down the
\keys{Shift} key and by clicking the right mouse button while
hovering the cursor over a directory. Select the \menu{Open Command
Window Here} option in the context menu that opens shortly
afterwards.

With online services -- such as Overleaf -- the commands are
executed automatically, although the locale may be erroneously
detected, or the \texttt{makeindex} tool (which is only able to
sort entries that contain digits and letters of the English
alphabet) may be used instead of \texttt{texindy}. In either case,
the index will be ill-sorted.

%\makeatletter\thesis@blocks@clear\makeatother
\phantomsection %% Print the index and insert it into the
\addcontentsline{toc}{chapter}{\indexname} %% table of contents.
\printindex

\appendix %% Start the appendices.
\chapter{An appendix}
Here you can insert the appendices of your thesis.

\end{document}
