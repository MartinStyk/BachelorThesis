\chapter{Prebalené APK súbory}
\label{Repackaged}
http://www.csc.ncsu.edu/faculty/jiang/pubs/CODASPY12.pdf
Pojem prebalený súbor označuje APK balíčky, ktoré boli modifikované no navonok sa prezentujú ako originálne neupravené aplikácie. Častým prípadom je, že takéto aplikácie pochádzajú z oficiálneho zdroja Android aplikácií – Google Play Strore, sú upravené a následne redistribuované pomocou neoficiálnych zdrojov. Takéto aplikácie si spravidla ponechávajú dizajn a funkcionalitu originálnych aplikácií, ku ktorej však môžu pridávať nové neželané funkcie alebo modifikácie. Hlavnou motiváciou pri modifikovaní aplikácií je šírenie škodlivého softvéru – malvéru. Pozmenená aplikácia môže napríklad získať prístup k citlivým informáciám uložených v Android zariadení alebo monitorovať správanie užívateľa. Modifikovaná aplikácia môže obsahovať nové reklamy. Prebalené aplikácie majú negatívny vplyv na vývojárov originálnej aplikácie. V prípade, že aplikácia obsahuje možnosť nákupov priamo z aplikácie\footnote{angl. in-app purchase}, výnosy z týchto predajov môžu byť presmerované z účtov originálnych vývojárov na účet ľudí, ktorí aplikáciu modifikovali. Ovplyvnené sú aj obchody s aplikáciami na ktorých sa nachádzajú prebalené aplikácie, keďže používatelia uprednostnia kvalitnejšie zdroje. V súčasnosti žiadny z obchodov s aplikáciami vrátane oficiálneho Google Play nepoužíva efektívnu detekciu prebalených aplikácií. http://www.cs.columbia.edu/~nieh/pubs/sigmetrics2014\_playdrone.pdf

\section{Modifikácia APK súborov}
Modifikácia APK súborov nie je náročná. Aplikácia môže byť jednoducho rozbalená a zdrojové súbory ako napríklad obrázky môžu byť upravené alebo nahradené inými.  Štrukturovaný proces tvory APK (http://www.alittlemadness.com/2010/06/07/understanding-the-android-build-process/) balíčkov umožňuje jednoduchú dekompiláciu. V prípade modifikácie zdrojového kódu je možné použiť existujúce nástroje ako \zv{dex2Jar} alebo \zv{ApkTool}, ktoré sú bližšie popísané v kapitole \ref{nastroje_revezneho_inzinierstva}. Modifikácia súboru \zv{AndroidManifest.xml} je takisto možná. Pomocou existujúcich nástrojov je ho možné previesť do čitateľného XML súboru, ktorý je možné editovať a následne previesť späť do binárneho XML formátu. Túto funkcionalitu poskytuje okrem iných utilít aj ApkTool.
Android obsahuje ochranu pred narušením integrity APK balíčka, ktorá je zabezpečená pomocou súborov v priečinku META-INF v koreňovej zložke APK súboru. Informácie o týchto súboroch sa nachádzajú v kapitole \ref{META-INF}. V prípade detekcii narušenia integrity, Android zakáže inštaláciu danej aplikácie. Po každej zmene súborov v APK balíčku je nutné podpísať ho. Aplikácie sú zvyčajne podpísané certifikátom ktorý je podpísaný identitou ktorú identifikuje\footnote{angl. self-signed certificate}. Vďaka tomu je možné po modifikácií APK balíček podpísať a zabezpečiť tak jeho fungovanie.

\section{Známe metódy detekcie prebalených APK súborov}

Jednoduchá modifikácia inštalačných balíčkov predstavuje problém pre celkový ekosystém aplikácií pre Android. Riešenie problému pozmeňovania a redistribúcie APK súborov je v súčasnosti dôležitou témou. Bolo navrhnutých viacero spôsobov detekcie prebalených aplikácií. 

\subsection{Detekcia pomocou analýzy zdrojového kódu}
Väčšina navrhnutých spôsobov detekcie modifikovaných APK balíčkov využíva metódu analyzujúcu zdrojový kód aplikácie spolu so súborom AndroidManifest.xml. Úprava zdrojového kódu je nevyhnutná v prípade editácie za účelom pridania novej neželanej funkcionality, pridania nových knižníc s reklamou alebo editácie pôvodnej reklamy použitej v aplikácii. Motiváciou pre editáciu metasúboru AndroidManifest.xml je možnosť pridávať aplikáciám prístupové povolenia (viď kapitola \ref{el_uses-permission}).
Riešenia sú založené na použití statickej analýzy kódu, dynamickej analýzy alebo na detekcií známych vzoriek škodlivého kódu\footnote{angl. signature based}. Takéto riešenia boli prezentované v prácach TODO.

Huang, H., Zhu, S., Liu, P., Wu, D.: A framework for evaluating mobile app repackaging detection algorithms. In: Proc. of TRUST ’13. pp. 169–186 (2013)
Crussell, J., Gibler, C., Chen, H.: Attack of the clones: Detecting cloned applications on android markets. In: Proc. of ESORICS’12. pp. 37–54 (2012)
Crussell, J., Gibler, C., Chen, H.: Scalable semantics-based detection of similar android applications. In: Proc. of Esorics’13 (2013)
Gibler, C., Stevens, R., Crussell, J., Chen, H., Zang, H., Choi, H.: Adrob: examining the landscape and impact of android application plagiarism. In: Proc. of MobiSys’13. pp. 431–444 (2013)
Hanna, S., Huang, L., Wu, E., Li, S., Chen, C., Song, D.: Juxtapp: a scalable system for detecting code reuse among android applications. In: Proc. of DIMVA’12. pp. 62–81 (2013)
Zhou, W., Zhou, Y., Jiang, X., Ning, P.: Detecting repackaged smartphone applications in third-party android marketplaces. In: Proc. of CODASPY’12. pp. 317–326 (2012)	
Potharaju, R., Newell, A., Nita-Rotaru, C., Zhang, X.: Plagiarizing smartphone applications: attack strategies and defense techniques. In: Proc. of ESSoS’12. pp. 106–120 (2012)
%http://dl.acm.org/citation.cfm?id=2046619
%http://web.engr.illinois.edu/~caesar/courses/CS598.S13/slides/philip_IDS_practice.pdf - signature based
	
\subsection{Detekcia pomocou podobnosti súborov}

Repackaged APK súbory je možné úspešne detekovať prostredníctvom zhody súborov obsiahnutých v APK balíčku. Tento prístup prezentoval Yury Zhauniarovich v svojej práci \zv{FSquaDRA: Fast Detection of Repackaged Applications} (http://www.zhauniarovich.com/files/pubs/Fsquadra\_Zhauniarovich2014.pdf , https://github.com/zyrikby/FSquaDRA). Prístup využíva skutočnosť, že aplikácia nie je definovaná iba svojim zdrojovým kódom a funkcionalitou, ale je tvorená aj ďalšími dôležitými prvkami ako sú používateľské prostredie alebo multimediálny obsah. APK balíčky obsahujú množstvo doplnkových zdrojových súborov.  Základom tohto prístupu je pozorovanie, že modifikované aplikácie zachovávajú užívateľské rozhranie ,dizajn, ikony, obrázky alebo zvuky pôvodných aplikácií. Práve tieto prvky výrazne odlišujú aplikácie, identifikujú ich pre užívateľov a majú výrazný dopad na užívateľský dojem. Preto je veľká časť súborov z neupravených APK balíčkoch obsiahnutá aj v modifikovaných balíčkoch. Originálna aplikácia Opera Mini a verzia tejto aplikácie obsahujúca malware (http://www.zdnet.com/
warning-new-android-malware-tricks-users-with-real-opera-mini-7000001586/), sa zhodujú v 230 z 234 súborov nachádzajúcich sa v príslušných APK balíčkoch. Riešenie prezentované v práci \zv{FSquaDRA} porovnáva všetky súbory medzi dvoma APK balíčkami.  Porovnávanie jednotlivých súborov na binárnej úrovni by bolo výpočtovo náročné. Preto sa na porovnanie využívajú SHA1 hashe súborov, ktoré sa nachádzajú v súbore \zv{MANIFEST.MF} (viď \ref{MANIFEST.MF}). Podobnosť aplikácií je určená na základe Jaccard indexu. V práci sa rozlišujú dva typy podobných APK súborov. Aplikácie sú považované za plagiátorsky prebalené aplikácie, keď obsahujú mnoho rovnakých súborov, ale sú podpísané rôznymi certifikátmi. V prípade veľkej zhody súborov a  identických certifikátov, sú aplikácie považované za rôzne verzie jednej aplikácie a nie sú označené ako nebezpečné. Tento spôsob porovnávania neumožňuje určiť, ktorá z aplikácií je originálna, a ktorá je pozmenená.  Umožňuje však rýchlu a efektívnu detekciu modifikovaných APK súborov. 

\section{Navrhnutá metóda detekcie prebalených APK súborov}

Spôsob detekcie pozmenených APK súborov prezentovaný v rámci tejto práce vychádza zo základných metód prezentovaných v článku \zv{FSquaDRA: Fast Detection of Repackaged Applications}. Prístup je založení na podobnosti súborov. Celková podobnosť aplikácií je určená podľa počtu zhodných súborov prítomných v oboch APK balíčkoch. Účelom našej implementácie nie je simulovať detekciu prebalených inštalačných súborov pomocou metódy \zv{FSquaDRA}. Cieľom je implementovať program, ktorý na detekciu modifikovaných APK balíčkov používa podobnosť obsahu balíčkov, ktorú kombinuje s metadátami a informáciami o daných APK súborov. Metadáta sú využívané na zefektívnenie výpočtu, ktoré je dosiahnuté neporovnávaním súborov medzi dvojicami zjavne odlišných aplikácií. Informácie získané porovnávaním a analýzou dvoch podobných aplikácií sú užívateľovi prezentované ako výstup porovnania. V prípade podobnosti aplikácií je výstupom porovnania zoznam odlišností, a typ podobnosti dvoch aplikácií. Typ podobnosti je určený na základe zhody certifikátov a zhody verzií aplikácií. 

\subsection{Implementácia}

Funkcionalita porovnávania a detekcie modifikovaných APK balíčkov je implementovaná v programe \zv{ApkAnalyzer}. \zv{ApkAnalyzer} je aplikácia napísaná v jazyku Java, ktorá neobsahuje grafické užívateľské rozhranie. Používateľ môže aplikáciu spúšťať a zadávať mu parametre pomocou príkazového riadku. Funkcionalita porovnávania APK súborov sa spúšťa pomocou parametra \zv{–compare}. Vstup pre porovnávanie APK súborov nie sú samotné APK balíčky, ale JSON súbory obsahujúce informácie o aplikáciách, ktoré sú vytvorené aplikáciou ApkAnalyzer počas analýzy APK balíčkov (viď kapitola \ref{analyza}). Analýza a následné porovnanie môžu byť spustené sériovo. Samotné porovnávanie prebieha párovo, každá aplikácia je porovnávaná so všetkými ostatnými. Proces porovnávania je paralelizovaný a každé dostupné procesorové jadro porovnáva inú dvojicu aplikácií. 
Porovnávanie a vyhodnocovanie podobnosti je rozdelené do viacerých etáp. Najskôr sa porovnávajú základné informácie o APK súboroch a príslušných aplikáciách. Toto porovnanie využíva základné metadata o aplikáciách a zahŕňa veľkosť APK súboru, počet komponent z ktorých sa aplikácia skladá (aktivity, služby, poskytovatelia obsahu, prijímače), počet rôznych obrázkových súborov a počet súborov definujúcich vzhľad obrazoviek(ang. layout). Všetky tieto hodnoty sú číselné. Je nutné aby implementácia ich porovnávania bola funkčná pre malé aj veľké hodnoty. Taktiež je nutné zabezpečiť komutatívnosť, ktorá zaručí, že nezáleží na poradí porovnávania aplikácií a teda func(A,B) = func(B,A). Získané hodnoty sú porovnávane s minimálnymi hodnotami potrebnými na to, aby boli aplikácie považované za podobné. Tieto hodnoty je možné meniť editáciou súboru similarity.properties v koreňovej zložke projektu \zv{ApkAnalyzer}.
V prípade detekcie základnej podobnosti sa porovnajú všetky súbory v APK balíčkoch. Podobne ako v aplikácií vyvinutej v rámci projektu \zv{FSquaDRA}, porovnávajú sa SHA1 hashe súborov uložené v \zv{MANIFEST.MF}. Separátne sú porovnávané súbory \zv{classes.dex}, \zv{resources.arsc}. Ostatné súbory sú porovnávané v rámci kategórií : súbory v priečinku drawable, súbory v priečinku layout, ostatné súbory, všetky súbory. Zhoda súborov medzi dvomi APK balíčkami je určená pomocou Jaccard index(https://en.wikipedia.org/wiki/Jaccard\_index). Nech A sú súbory v danej kategórií v jednom APK balíčku, a B sú súbory v danej kategórií v druhom porovnávanom APK balíčku. Jaccard Index(A,B) = A prienik B / A zjednotenie B. Aplikácie sú považované za podobné v prípade, že Jaccard index pre každú z kategórií prekračuje minimálnu hodnotu definovanú v súbore \zv{similarity.properties}. Okrem súborov sa porovnajú aj všetky hodnoty získané analýzou APK súboru, no určenie podobnosti v tejto fáze prebieha len na základe rovnakých súborov. 

\paragraph{Typ podobnosti}\mbox{}\\
Zhoda certifikátov a zhoda verzií aplikácií je vypočítaná za účelom určenia typu podobnosti daných aplikácií. Pre zhodu verzií a certifikátov rozlišuje tri hodnoty – rovnaké, rozdielne alebo neurčené. Hodnota neurčené je použitá v prípade, že sa dáta nepodarilo získať. Zhoda certifikátov sa určuje na základe MD5 hashu certifikátu. Tento údaj je v kontexte detekcie modifikovaných APK súborov veľmi dôležitý. V prípade zhody certifikátov je zaručené, že APK súbory pochádzajú od rovnakého vydavateľa. Pokiaľ sú certifikáty rozdielne, pôvodca súborov je z najväčšou pravdepodobnosťou rozdielny. Zhoda verzií aplikácií je využitá na detekciu rovnakých aplikácií v rozdielnych verziách.
 Kombinácia týchto hodnôt určuje 9 kategórií podobnosti APK súborov. Každá z týchto kategórií napovedá a vzájomnom vzťahu danej dvojice Android aplikácií. Najväčšia pravdepodobnosť, že aplikácia je prebalená je v prípade rovnakých verzií a zároveň rozdielnych certifikátov. 

\paragraph{Výstup porovnania}\mbox{}\\
V prípade, že porovnávaná dvojica APK súborov je vyhodnotená ako podobná, \zv{ApkAnalyzer} vytvorí výstupný súbor vo formáte JSON obsahujúci rozdiely medzi danými aplikáciami.  Tento súbor obsahuje rozdiely určené na základe metadát a porovnania aplikácií. Slúži ako jednoduchá obdoba linuxového príkazu diff implementovaná nad APK súbormi. Obsahuje informácie o modifikovaných parametroch a komponentoch aplikácií a taktiež zoznam upravených, nových alebo odstránených súborov.  Príklad takéhoto výstupného súboru sa nachádza v prílohe TODO.
