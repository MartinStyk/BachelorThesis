\chapter{Analýza APK súborov}
\label{analyza}
Hlavnou úlohou práce je získať informácie o~APK súboroch ich detailnou analýzou. APK súbory majú pevnú štruktúru a jednoduchý formát, vďaka čomu je možná ich analýza a reverzné inžinierstvo. Reverzné inžinierstvo je proces analýzy funkcionality a obsahu aplikácie. Keďže APK súbory využívajú ZIP formát, mnohé informácie je možné získať jednoduchým rozbalením. Základnou úlohou analýzy a reverzného inžinierstva APK súborov v~tejto práci je získanie metadát o~APK súbore, ktoré sú využívané v~kapitole \ref{statistiky} a \ref{Repackaged}.

\section{Nástroje reverzného inžinierstva}
\label{nastroje_revezneho_inzinierstva}

Existuje viacero nástrojov poskytujúcich funkcionalitu pre reverzné inžinierstvo Android aplikácií. Okrem aplikácií tretích strán je možné vo veľkej miere použiť aj nástroje obsiahnuté v~\zv{Android Software Development Kit (SDK)}. \zv{Android SDK} je kolekcia štandardných nástrojov používaných pri vývoji a zostavení Android aplikácií. 

\subsection{ApkTool}
\label{ApkTool}
\zv{ApkTool} je nástroj na reverzné inžinierstvo Android aplikácií. Dokáže dekódovať zdroje aplikácie do takmer originálnej podoby. Do čitateľnej podoby prevádza súbory \zv{resources.arsc}, \zv{classes.dex} aj binárne XML súbory. Z~dekódovaných súborov umožňuje opätovné zostavenie APK súboru. Súbor \zv{classes.dex} je dekompilovaný do súborov vo formáte \zv{smali}~\cite{Nolan2012a}. Takéto súbory obsahujú nízkoúrovňový zdrojový kód. \zv{ApkTool} podporuje debugovanie \zv{smali} kódu~\cite{apkTool}.

\subsection{Dex2Jar}
\label{Dex2Jar}
Nástroj \zv{Dex2Jar} podporuje dekódovanie DEX súborov do formátu skompilovaných CLASS súborov .Výsledné CLASS súbory môžu byť prevedené do čitateľného kódu v~jazyku Java pomocou dekompilátoru \zv{JD-GUI}. Pracuje výhradne so súborom \zv{classes.dex} a nepodporuje prevod binárnych XML do čitateľnej podoby.

\subsection{AAPT}
\label{AAPT}

\zv{Android Asset Packaging Tool} (\zv{AAPT}) je štandardný nástroj obsiahnutý v~\zv{Android SDK}. Nástroj \zv{AAPT} umožňuje vytvorenie, aktualizovanie a prezeranie súborov vo formáte APK. Dokáže skompilovať zdrojové súbory do binárnej formy a umožňuje aj ich čiastočnú dekompiláciu\cite{aapt}.

\subsection{AXML}
\label{AXML}
\zv{AXML} je knižnica navrhnutá na prácu s~binárnymi XML súbormi, ktoré vznikajú počas zostavenia Android aplikácie pomocou nástroja \zv{AAPT}. Knižnica umožňuje prevod takýchto XML súborov do čitateľného XML formátu, je implementovaná v~jazyku Java.

\section{Implementácia analýzy}
Analýza APK súborov je implementovaná v~rámci programu \zv{ApkAnalyzer} a môže byť spustená pomocou argumentu \zv{–analyze}. Zároveň je potrebné špecifikovať analyzovaný APK súbor alebo priečinok obsahujúci takéto súbory pomocou argumentu \zv{–in} a priečinok do ktorého bude zapísaný výstup analýzy pomocou argumentu \zv{–out}. \zv{ApkAnalyzer} je aplikácia prispôsobená na prácu s~veľkým počtom APK súborov. Proces analýzy je preto paralelizovaný a každé dostupné procesorové jadro analyzuje inú aplikáciu. Počas analýzy využíva \zv{ApkAnalyzer} nástroj na reverzné inžinierstvo \zv{ApkTool}, ktorý slúži na dekompiláciu zdrojových súborov, ktoré sú v~APK balíčku uložené v~nečitateľnej forme. Pri analýze APK súborov nevyužívame dekompilovaný zdrojový kód vo formáte \zv{smali} a preto je dekompilácia súboru \zv{classes.dex} vynechaná. \zv{ApkTool} je samostatná aplikácia, avšak \zv{ApkAnalyzer} ju nespúšta prostredníctvom príkazového riadka, ale používa ju ako knižnicu a pristupuje priamo k~požadovaným funkciám. V~programe je využitá aj knižnica \zv{AXML}. Za účelom analýzy je každý APK súbor rozbalený štandardným spôsobom ako ZIP archív a dekompilovaný pomocou nástroja \zv{ApkTool}. Proces analýzy jednej aplikácie vyžaduje viacero prístupov na pevný disk, čo má negatívny vplyv na čas analýzy. Aplikácia je rozbalená obidvoma spôsobmi do dočasného adresára. Následne sa pristupuje k~jednotlivým súborom a na záver je celý dočasný adresár zmazaný.

\subsection{Zbierané metadáta}
Nástroj \zv{ApkAnalyzer} implementuje zbieranie informácií z~rôznych častí APK súboru. Zbierané metadáta je možné rozdeliť do piatich základných kategórií:
\begin{itemize}
\item základné informácie o~súbore,
\item informácie zo súboru \zv{AndroidManifest.xml},
\item informácie o~certifikáte,
\item informácie o~zdrojových súboroch\footnote{angl. resources},
\item súbory obsiahnuté v~APK balíčku.
\end{itemize}

\noindent Kompletný zoznam zbieraných metadát sa nachádza v~prílohe \ref{tab:zbieraneData}.

\subsection*{Základné informácie o~súbore} 
V~tejto kategórii sa nachádzajú informácie ako je veľkosť APK súboru alebo veľkosti súborov \zv{classes.dex} a \zv{resources.arsc}. Pre získanie veľkosti súborov obsiahnutých v~APK balíčku je balíček rozbalený do dočasného adresára. Získaná veľkosť APK súboru reprezentuje veľkosť skomprimovaného archívu. Hodnoty veľkosti jednotlivých súborov predstavujú ich veľkosť po dekomprimácii. Táto kategória obsahuje aj samotný názov súboru. 
\subsection*{Informácie zo súboru \zv{AndroidManifest.xml}} 
\zv{AndroidManifest.xml} predstavuje hlavný zdroj meta informácií o~aplikácii pre systém Android (viď \ref{AndroidManifest.xml}). Dáta nachádzajúce sa v~tomto súbore tvoria významnú časť dát získaných našou analýzou. Na prevod z~binárneho XML formátu je primárne použitá knižnica \zv{AXML} (viď \ref{AXML}), v~prípade zlyhania konverzie sa použije  nástroj \zv{ApkTool} (viď \ref{ApkTool}). Dáta získané analýzou tohto súboru zahŕňajú napríklad verziu aplikácie, použité prístupové oprávnenia alebo komponenty z~ktorých sa aplikácia skladá.
\subsection*{Informácie o~certifikáte}
Táto kategória obsahuje dáta získané zo súboru \zv{CERT.RSA} v~priečinku \zv{META-INF} (viď \ref{CERT.RSA}). Obsahujú napríklad použitý algoritmus podpisovania, názov vydavateľa alebo MD5 hash celého certifikátu. Pred prístupom k~súboru \zv{CERT.RSA}, ktorý obsahuje X509 certifikát je nutné APK balíček rozbaliť. Na analýzu certifikátu je použité API obsiahnuté v~štandardnom java balíčku \zv{java.security.cert}.
\subsection*{Informácie o~zdrojových súboroch\footnote{angl. resources}} 
Kategória obsahujúca informácie o~zdrojoch aplikácie, napríklad formát alebo veľkosť obrázkových súborov, počet lokalizácií aplikácie alebo počet surových neskomprimovaných zdrojových súborov. Pre získanie metadát tohto typu boli XML súbory prevedené do čitateľného formátu. Na základe informácií o~štruktúre priečinku \path{res} (viď \ref{res}) bol určený počet zdrojov a ich typ.
\subsection*{Súbory obsiahnuté v~APK balíčku}
Zoznam všetkých súborov rozdelený do kategórií: obrázky (súbory z~priečinku \cesta{res/drawable}), návrhy obrazoviek (súbory z~priečinku \cesta{res/layout}), \zv{classes.dex}, \zv{resources.arsc} a ostatné. O~každom súbore si uchovávame jeho relatívnu cestu v~APK balíčku a SHA1 hash. Ako zdroj informácií slúži súbor \zv{MANIFEST.MF} (viď \ref{MANIFEST.MF}).


\subsection{Výstup analýzy}
Pre každú analyzovanú aplikáciu je vygenerovaný výstupný súbor vo formáte JSON, obsahujúci získané metadáta o~danej aplikácii. Použitie súborov vo formáte JSON bolo zvolené z~dôvodu, že výstup získaný analýzou je ďalej spracovávaný v~ďalších častiach bakalárskej práce (viď kapitoly \ref{statistiky} a \ref{Repackaged}). JSON súbory sa dajú efektívne využiť na perzistentné ukladanie serializovaných Java objektov. Aplikácia \zv{ApkAnalyzer} využíva knižnicu \zv{gson} od spoločnosti \zv{Google}, ktorá poskytuje serializáciu objektu do podoby JSON súboru, a zároveň aj deserializáciu súboru na inštanciu príslušnej triedy. JSON súbory sú štrukturované a ľudsky čitateľné. Existuje viacero nástrojov na ich prehľadné hierarchické zobrazenie.  