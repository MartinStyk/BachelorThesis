\chapter{Analýza APK súborov}
https://www.rsaconference.com/writable/presentations/file\_upload/stu-w02b-beginners-guide-to-reverse-engineering-android-apps.pdf
http://www.sans.org/reading-room/whitepapers/pda/reverse-engineering-malware-android-33769
Hlavnou úlohou práce je získať informácie o APK súboroch ich detailnou analýzou. APK súbory majú pevnú štruktúru a jednoduchý formát, vďaka čomu je možná ich analýza a reverzné inžinierstvo. Reverzné inžinierstvo je proces analýzy funkcionality a obsahu aplikácie. Keďže APK súbory využívajú ZIP formát, mnohé informácie je možné získať jednoduchým rozbalením. Základnou úlohou analýzy je získanie metadát o APK súbore.

\section{Nástroje reverzného inžinierstva}

Existuje viacero nástrojov poskytujúcich funkcionalitu pre reverzné inžinierstvo Android aplikácií. Okrem aplikácií tretích strán je možné vo veľkej miere použiť aj štandardné nástroje obsiahnuté v Android Software Development Kit.

\subsection{ApkTool}
\label{ApkTool}
 (http://ibotpeaches.github.io/Apktool/)
Nástroj na reverzné inžinierstvo Android aplikácií. Dokáže dekódovať zdroje aplikácie do takmer originálnej podoby. Do čitateľnej podoby prevádza súbory resources.arsc, classes.dex aj binárne XML súbory. Z dekódovaných súborov umožňuje opätovné zostavenie APK súboru. Súbor classes.dex je dekompilovaný do súborov vo formáte SMALI. Smali súbory obsahujú nízko úrovňový kód na úrovni asembleru. ApkTool podporuje debugovanie smali kódu.

\subsection{Dex2Jar}
(https://github.com/pxb1988/dex2jar) http://pof.eslack.org/2011/02/18/from-apk-to-readable-java-source-code-in-3-easy-steps/
Nástroj podporujúci dekódovanie DEX súborov do formátu skompilovaných CLASS súborov http://fileinfo.com/extension/class .Výsledné CLASS súbory môžu byť prevedené do čitateľného kódu v jazyku Java pomocou dekompilátoru JD-GUI(http://jd.benow.ca/ ) Pracuje výhradne so súborom classes.dex a nepodporuje prevod binárnych XML do čitateľnej podoby.

\subsection{AXML}
\label{AXML}
https://github.com/xgouchet/AXML
AXML je knižnica navrhnutá na prácu s binárnymi XML súbormi, ktoré vznikajú počas zostavenia Android aplikácie pomocou nástroja AAPT. Knižnica umožňuje prevod takýchto XML súborov do čitateľného XML formátu, je implementovaná v jazyku Java.


\subsection{AAPT}
\label{AAPT}
http://elinux.org/Android\_aapt
Android Asset Packaging Tool (AAPT) je štandardný nástroj obsiahnutý v Android Software Development Kit (SDK). Android SDK je kolekcia nástrojov používaných pri vývoji a zostavení Android aplikácií. Nástroj AAPT umožňuje vytvorenie, aktualizovanie a prezeranie súborov vo formáte APK. Dokáže skompilovať zdrojové súbory do binárnej formy a umožňuje aj ich dekompiláciu.

\section{Implementácia analýzy}
Analýza APK súborov je implementovaná v rámci programu ApkAnalyzer a môže byť spustená pomocou argumentu –analyze. Zároveň je potrebné špecifikovať analyzovaný APK súbor alebo priečinok obsahujúci takéto súbory pomocou argumentu –in a priečinok do ktorého bude zapísaný výstup analýzy pomocou argumentu –out. ApkAnalyzer je aplikácia prispôsobená na prácu s veľkým počtom APK súborov, proces analýzy je preto paralelizovaný a každé dostupné procesorové jadro analyzuje inú aplikáciu. Pre každú analyzovanú aplikáciu je vygenerovaný výstupný súbor vo formáte JSON obsahujúci získané metadáta o danej aplikácií. \\\\
Zbierané metadáta je možné rozdeliť do piatich kategórií:
\begin{itemize}
\item Základné informácie o APK súbore -- v tejto kategórií sa nachádzajú informácie ako je veľkosť APK súboru alebo súborov \zv{classes.dex} a \zv{resources.arsc}. Pre získanie veľkosti súborov obsiahnutých v APK balíčku je balíček rozbalený do dočasného adresára
\item Informácie zo súboru \zv{AndroidManifest.xml} -- \zv{AndroidManifest.xml} predstavuje hlavný zdroj meta informácií o aplikácii pre systém Android(viď \ref{AndroidManifest.xml}). Dáta nachádzajúce sa v tomto súbore tvoria významnú časť dát získaných našou analýzou. Na prevod z binárneho XML formátu je primárne použitá knižnica AXML (viď \ref{AXML}), v prípade, že prevod zlyhá, použije sa nástroj ApkTool (viď \ref{ApkTool}). Dáta analýzou tohto súboru zahŕňajú napríklad verziu aplikácie, použité prístupové oprávnenia alebo komponenty z ktorých sa aplikácia skladá
\item Informácie o certifikáte -- dáta získané analýzou súboru \zv{CERT.RSA} v priečinku \zv{META-INF}(viď \ref{CERT.RSA}), napríklad použitý algoritmus podpisovania, názov vydavateľa alebo MD5 hash celého certifikátu. Pred prístupom k súboru je \zv{CERT.RSA} je nutné APK balíček rozbaliť
\item Informácie o zdrojových súboroch\footnote{angl. resources} -- informácie o zdrojoch aplikácie, ako napríklad formát alebo veľkosť obrázkových súborov, počet lokalizácií aplikácie alebo počet surových neskomprimovaných zdrojových súborov
\item Súbory obsiahnuté v APK balíčku -- zoznam všetkých súborov rozdelený do kategórií: obrázky(súbory z priečinku \cesta{res/drawable}), návrhy obrazoviek(súbory z priečinku \cesta{res/layout}), \zv{classes.dex}, \zv{resources.arsc} a ostatné. O každom súbore si uchovávame jeho relatívnu cestu v APK balíčku a SHA1 hash. Ako zdroj informácií slúži súbor \zv{MANIFEST.MF} (viď \ref{MANIFEST.MF})
\end{itemize}

\noindent Kompletný zoznam zbieraných metadát sa nachádza v prílohe TODO.  