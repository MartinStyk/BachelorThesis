\chapter{Analýza APK súborov}
\label{analyza}
Hlavnou úlohou práce je získať informácie o~APK súboroch ich detailnou analýzou. APK súbory majú pevnú štruktúru a jednoduchý formát, vďaka čomu je možná ich analýza a reverzné inžinierstvo. Reverzné inžinierstvo je proces analýzy funkcionality a obsahu aplikácie. Keďže APK súbory využívajú ZIP formát, mnohé informácie je možné získať jednoduchým rozbalením. Základnou úlohou analýzy a reverzného inžinierstva APK súborov v~tejto práci je získanie metadát o~APK súbore, ktoré sú využívané v~kapitole \ref{statistiky} a \ref{Repackaged}.

\section{Nástroje reverzného inžinierstva}
\label{nastroje_revezneho_inzinierstva}

Existuje viacero nástrojov poskytujúcich funkcionalitu pre reverzné inžinierstvo Android aplikácií. Okrem aplikácií tretích strán je možné vo veľkej miere použiť aj nástroje obsiahnuté v~\zv{Android Software Development Kit (SDK)}.\zv{Android SDK} je kolekcia štandardných nástrojov používaných pri vývoji a zostavení Android aplikácií. 

\subsection{ApkTool}
\label{ApkTool}
Nástroj na reverzné inžinierstvo Android aplikácií. Dokáže dekódovať zdroje aplikácie do takmer originálnej podoby. Do čitateľnej podoby prevádza súbory \zv{resources.arsc}, \zv{classes.dex} aj binárne XML súbory. Z~dekódovaných súborov umožňuje opätovné zostavenie APK súboru. Súbor \zv{classes.dex} je dekompilovaný do súborov vo formáte SMALI. Smali súbory obsahujú nízkoúrovňový kód na úrovni asembleru. ApkTool podporuje debugovanie smali kódu~\cite{apkTool}.

\subsection{Dex2Jar}
\label{Dex2Jar}
Nástroj podporujúci dekódovanie DEX súborov do formátu skompilovaných CLASS súborov .Výsledné CLASS súbory môžu byť prevedené do čitateľného kódu v~jazyku Java pomocou dekompilátoru \zv{JD-GUI}. Pracuje výhradne so súborom \zv{classes.dex} a nepodporuje prevod binárnych XML do čitateľnej podoby.

\subsection{AXML}
\label{AXML}
\zv{AXML} je knižnica navrhnutá na prácu s~binárnymi XML súbormi, ktoré vznikajú počas zostavenia Android aplikácie pomocou nástroja \zv{AAPT}. Knižnica umožňuje prevod takýchto XML súborov do čitateľného XML formátu, je implementovaná v~jazyku Java.


\subsection{AAPT}
\label{AAPT}

\zv{Android Asset Packaging Tool} (\zv{AAPT}) je štandardný nástroj obsiahnutý v~\zv{Android SDK}. Nástroj \zv{AAPT} umožňuje vytvorenie, aktualizovanie a prezeranie súborov vo formáte APK. Dokáže skompilovať zdrojové súbory do binárnej formy a umožňuje aj ich dekompiláciu\cite{aapt}.

\section{Implementácia analýzy}
Analýza APK súborov je implementovaná v~rámci programu \zv{ApkAnalyzer} a môže byť spustená pomocou argumentu \zv{–analyze}. Zároveň je potrebné špecifikovať analyzovaný APK súbor alebo priečinok obsahujúci takéto súbory pomocou argumentu \zv{–in} a priečinok do ktorého bude zapísaný výstup analýzy pomocou argumentu \zv{–out}. \zv{ApkAnalyzer} je aplikácia prispôsobená na prácu s~veľkým počtom APK súborov. Proces analýzy je preto paralelizovaný a každé dostupné procesorové jadro analyzuje inú aplikáciu. Pre každú analyzovanú aplikáciu je vygenerovaný výstupný súbor vo formáte JSON obsahujúci získané metadáta o~danej aplikácií. \\\\
Zbierané metadáta je možné rozdeliť do piatich kategórií:
\begin{itemize}
\item Základné informácie o~APK súbore -- v~tejto kategórií sa nachádzajú informácie ako je veľkosť APK súboru alebo veľkosti súborov \zv{classes.dex} a \zv{resources.arsc}. Pre získanie veľkosti súborov obsiahnutých v~APK balíčku je balíček rozbalený do dočasného adresára
\item Informácie zo súboru \zv{AndroidManifest.xml} -- \zv{AndroidManifest.xml} predstavuje hlavný zdroj meta informácií o~aplikácii pre systém Android (viď \ref{AndroidManifest.xml}). Dáta nachádzajúce sa v~tomto súbore tvoria významnú časť dát získaných našou analýzou. Na prevod z~binárneho XML formátu je primárne použitá knižnica \zv{AXML} (viď \ref{AXML}), v~prípade zlyhania konverzie sa použije  nástroj \zv{ApkTool} (viď \ref{ApkTool}). Dáta získané analýzou tohto súboru zahŕňajú napríklad verziu aplikácie, použité prístupové oprávnenia alebo komponenty z~ktorých sa aplikácia skladá
\item Informácie o~certifikáte -- dáta získané zo súboru \zv{CERT.RSA} v~priečinku \zv{META-INF} (viď \ref{CERT.RSA}). Obsahujú napríklad použitý algoritmus podpisovania, názov vydavateľa alebo MD5 hash celého certifikátu. Pred prístupom k~súboru \zv{CERT.RSA} je nutné APK balíček rozbaliť
\item Informácie o~zdrojových súboroch\footnote{angl. resources} -- informácie o~zdrojoch aplikácie, napríklad formát alebo veľkosť obrázkových súborov, počet lokalizácií aplikácie alebo počet surových neskomprimovaných zdrojových súborov
\item Súbory obsiahnuté v~APK balíčku -- zoznam všetkých súborov rozdelený do kategórií: obrázky(súbory z~priečinku \cesta{res/drawable}), návrhy obrazoviek(súbory z~priečinku \cesta{res/layout}), \zv{classes.dex}, \zv{resources.arsc} a ostatné. O~každom súbore si uchovávame jeho relatívnu cestu v~APK balíčku a SHA1 hash. Ako zdroj informácií slúži súbor \zv{MANIFEST.MF} (viď \ref{MANIFEST.MF})
\end{itemize}

\noindent Kompletný zoznam zbieraných metadát sa nachádza v~prílohe \ref{tab:zbieraneData}.