\chapter{Štatistiky}
\label{statistiky}
Analýzou jednotlivých APK súborov získame detailné informácie o~jednotlivých aplikáciách. Pre ucelenejší pohľad na všeobecné vlastnosti a atribúty Android aplikácií je vhodné rozšíriť analýzu jednotlivých aplikácií na skúmanie väčšej množiny APK balíčkov. Keďže databáza APK súborov vytvorená v~tejto práci obsahuje dostatočne veľkú vzorku približne 20000 APK súborov, ktoré pochádzajú z~rôznych oficiálnych aj alternatívnych zdrojov, poskytuje dobrú vzorku na určenie štatistických údajov o~Android aplikáciách. Štatistické informácie prezentované v~tejto kapitole sa viažu k~aplikáciám dostupným v~rokoch 2014--2016, a teda sú aktuálne pre spomenuté obdobie.

Vyvinutá aplikácia \zv{ApkAnalyzer} poskytuje možnosť výpočtu štatistík nad množinou APK súborov. Funkcionalita výpočtu štatistických informácií sa aktivuje pomocou prepínača \zv{–statistics} pri spustení programu z~príkazového riadku. Ako vstup aplikácie slúžia JSON súbory vytvorené analýzou popísanou v~kapitole \ref{analyza}. Výstupom je súbor vo formáte JSON obsahujúci vypočítané štatistické dáta. Pri vlastnostiach, ktorých hodnota je vyjadrená číselne sú vypočítané základné matematické štatistiky ako je aritmetický priemer, modus, medián, rozptyl, smerodajná odchýlka, minimum a maximum. Pri najnižšej a najvyššej hodnote obsahuje výstup aj názov aplikácií, ktoré tieto hodnoty dosahujú. V~prípade vlastností, ktoré nadobúdajú obmedzený počet vopred definovaných hodnôt je určené percentuálne zastúpenie jednotlivých hodnôt.

\section{Získané dáta}
\subsection*{Veľkosť APK súborov}
Analýzou databázy APK súborov sa zistilo, že medián veľkosti APK súborov je 5,26\,MB, aritmetický priemer dosahuje hodnotu 10,19\,MB. Keďže APK súbory využívajú ZIP formát, uvedené hodnoty reprezentujú veľkosť APK súboru ako skomprimovaného súboru. Po dekomprimácií je súčet veľkostí jednotlivých súborov spravidla väčší ako veľkosť APK súboru. Aritmetický priemer je štatistický ukazovateľ, ktorý je citlivý na extrémne hodnoty. Keďže je jeho hodnota v~porovnaní s~mediánom takmer dvojnásobná, v~našej databáze sa nachádzajú aplikácie, ktorých veľkosť výrazne prevyšuje najčastajšie a najbežnejšie veľkosti APK súborov.\\ Medián veľkosti súboru \zv{classes.dex} (viď. \ref{classes.dex}) je 2,16\,MB, aritmetický priemer je o~málo vyšší -- 2,86\,MB. Pri súbore \zv{resources.arsc} (viď \ref{resources.arsc}) bol zistený priemer veľkosti 690\,KB a medián 192\,KB. Uvedené hodnoty sú určené z~veľkosti súborov z~APK balíčku po ich dekompresii.

\subsection*{Počet súborov v~APK balíku}
Medián celkového počtu súborov v~APK balíčku je 397, aritmetický priemer má hodnotu približne 730 súborov.

\subsection*{Inštalačná politika}
Android poskytuje aplikáciám možnosť špecifikácie preferovaného pamäťového priestoru (interná alebo externá pamäť) a prípadnú možnosť presunutia nainštalovanej aplikácie (viď \ref{el_manifest}). Až 49\,\% aplikácií neumožňuje inštaláciu alebo presun na externé pamäťové médiá.  40\,\% aplikácií preferuje inštaláciu na interné úložisko s~možnosťou presunu do externej pamäte. 10,5\,\% aplikácií uprednostňuje inštaláciu na externé pamäťové médium. Rozdelenie hodnôt je zobrazené v~grafe na obrázku \ref{installLocFig}.

\begin{figure}[htb]
\centering
\begin{tikzpicture}[scale=0.8,align=center]
        \pie[text=label,rotate=240,
             explode=0,
             color={blue!70,cyan!70,red!70,orange!50}]{49.03/internalOnly, 40.47/auto, 10.5/preferExternal}
    \end{tikzpicture}
\caption{Hodnoty atribútu \zv{android:installLocation}}
\label{installLocFig}
\end{figure}

\subsection*{Prístupové oprávnenia}

Aplikácie najčastejšie deklarujú využívanie 4 prístupových oprávnení (viď \ref{el_uses-permission}). Medián počtu vyžadovaných oprávnení je 8. Najčastejšie využívané oprávnenia spolu s~ich percentuálnym zastúpením v~analyzovanej vzorke aplikácií sú uvedené v~tabuľke \ref{tab:permissions}. 
\begin{table}[!htbp]
\centering
  \begin{tabular}{|l r|}
    \hline
    Názov & \% \\\hline\hline
    android.permission.internet & 92,9 \\
    android.permission.access\_network\_state & 87,9 \\
    android.permission.write\_external\_storage & 75,2 \\
    android.permission.wake\_lock & 49,5 \\
    android.permission.read\_phone\_state & 49,4 \\
    android.permission.access\_wifi\_state & 44,7 \\
    android.permission.vibrate & 43,6 \\
    android.permission.get\_accounts & 31,3 \\
    android.permission.receive\_boot\_completed & 30,5 \\
    android.permission.vending.billing & 27,1 \\
    android.permission.access\_coarse\_location & 25,4 \\
    android.permission.access\_fine\_location & 25,4 \\
    com.google.android.c2dm.permission.receive & 22,0 \\
    android.permission.get\_tasks & 18,8 \\
    android.permission.read\_contacts & 18,5 \\
    android.permission.camera & 17,8 \\
    android.permission.write\_settings & 16,3 \\
    android.permission.read\_external\_storage & 16,2 \\
    android.permission.change\_wifi\_state & 12,9 \\
    android.permission.record\_audio & 12,8 \\
     android.permission.system\_alert\_window & 12,6 \\
    \hline
  \end{tabular}
  \caption{Najpoužívanejšie prístupové oprávnenia}
  \label{tab:permissions}
\end{table}

\subsection*{Komponenty aplikácií}

Základnou funkčnou jednotkou aplikácie je aktivita (viď \ref{el_activity}).Medián počtu aktivít medzi analyzovanými aplikáciami je  10, priemer dosahuje hodnotu 20,23. Aplikácie obsahujú najčastejšie 2 aktivity.\\Priemerný počet služieb definovaných v~aplikácii je 3,99, medián je 1, no najčastejším prípadom je, že aplikácia nedefinuje žiadnu službu.

\subsection*{Využité vlastnosti}
Aplikácie deklarujú nízky počet využívaných vlastností (viď \ref{el_uses-permission}). Aritmetický priemer je 1,44, medián a a modus sú nulové. Najčastejšie deklarované je využívanie vlastností uvedených v~tabuľke \ref{tab:features}.
\begin{table}[htb]
\centering
  \begin{tabular}{|l r|}
    \hline
    \textbf{Názov} & \textbf{\%} \\\hline\hline
    android.hardware.camera & 18,1 \\
    android.hardware.touchscreen & 16,1 \\
    android.hardware.telephony & 14,8 \\
    android.hardware.camera.autofocus & 10,6 \\
    android.hardware.location.gps & 10,2 \\
    android.hardware.location & 8,8 \\
    android.hardware.wifi & 8,4 \\
    android.hardware.location.network & 7,0\\
    android.hardware.bluetooth & 6,6\\
    android.hardware.touchscreen.multitouch & 6,0\\
    android.hardware.microphone & 5,2\\
    android.hardware.touchscreen.multitouch.distinct & 6,0\\
    android.hardware.touchscreen.screen.portrait & 4,4\\
    android.hardware.sensor.accelerometer & 3,9\\
    android.hardware.camera.flash & 3,7\\
    \hline
  \end{tabular}
  \caption{Najpoužívanejšie vlastnosti}
  \label{tab:features}
\end{table}

\subsection*{Podpis APK balíčka}
Na podpisovanie APK balíčkov je najčastejšie využitý algoritmus \zv{SHA1withRSA}, ktorý využíva až 74,87\,\% aplikácií. 15,56\,\% aplikácií je podpísaných pomocou algoritmu \zv{SHA256withRSA}, podpis pomocou \zv{MD5withRSA} je využitý v~5,88\,\% prípadoch. Percentuálne rozdelenie algoritmov použitých na podpis APK súborov je znázornený na obrázku \ref{fig:signAlg}.


\begin{figure}[htb]
\centering
\begin{tikzpicture}[scale=0.7,align=center]
        \pie[text=label,rotate=240,
             explode=0.2,
             color={blue!70,cyan!70,red!70,orange!50}]{74.9/SHA1withRSA, 15.6/SHA256withRSA, 5.9/MD5withRSA, 3.6/SHA1withDSA}
    \end{tikzpicture}
\caption{Algoritmus podpisu APK balíčku}
\label{fig:signAlg}
\end{figure}

\subsection*{Obrázkové súbory}
Analýza ukázala, že aplikácie využívajú množstvo obrázkových súborov. Medián celkového počtu obrázkových súborov v~APK balíčku je 210, aritmetický priemer dosahuje hodnotu 462 a najčastejším počtom obrázkových súborov je 5. Medián počtu rozdielnych obrázkov (veľkosť a rozlíšenie sa neberie do úvahy) je 134. Najčastejším formátom je PNG, aplikácia obsahuje priemerne 343 takýchto súborov.



\subsection*{Lokalizácia}
Aplikácie sú okrem základného jazyka lokalizované najčastejšie v~17 iných jazykoch. Aritmetický priemer počtu lokalizácií je 31,42. Lokalizácie sú ovplyvnené tým, že časť aplikácií bola stiahnutá z~portálov určených pre strednú Európu. V~českom jazyku je lokalizovaných 49\,\% aplikácií, v~slovenčine 46\,\%. Najčastejšie lokalizácie aplikácií sú uvedené v~tabuľke \ref{tab:language}.
\begin{table}[htb]
\centering
  \begin{tabular}{|l l r|}
    \hline
    \textbf{Kód} & \textbf{Jazyk} &  \textbf{\%} \\\hline\hline
    es & španielsky & 61,7 \\
    de & nemecký & 59,6 \\
    fr & francúzsky & 59,4 \\
    ru & ruský & 58,1 \\
    ja & japonský & 57,6 \\
    it & taliansky & 57,4 \\
	ko & korejský & 56,9 \\
	zh-rcn & čínsky (zjednodušený) & 55,6\\
	zh-rtw & čínsky (tradičný)& 54,0\\
	pt & portugalský & 52,6\\
	pl & polský & 51,7\\
	nl & holandský & 50,9\\
	tr & turecký & 50,6\\
	cs & český & 49,4\\
	sv & švédsky & 48,4\\
    \hline
  \end{tabular}
  \caption{Lokalizácia aplikácií}
  \label{tab:language}
\end{table}

\subsection*{Verzie Android SDK}

Najčastejšou najnižšou vyžadovanou verziou Android SDK je verzia 9 s~21,3\% zastúpením. Táto verzia je asociovaná so systémom Android 2.3 až Android 2.3.2 \zv{Gingerbread} a bola vydaná v~novembri 2010. Nanižšie vyžadované verzie Android SDK v~našej databáze APK súborov sú zobrazené v~grafe \ref{tab:minSdk}  a graf na obrázku \ref{minSdkGraf}.
\begin{figure}[htb]
\centering
\begin{bchart}[min=0,max=25,step=5,unit=\%]
\bcbar[label=SDK 11]{2.09}
\bcskip{3pt}
\bcbar[label=SDK 5]{3.65}
\bcskip{3pt}
\bcbar[label=SDK 15]{3.72}
\bcskip{3pt}
\bcbar[label=SDK 3]{5.57}
\bcskip{3pt}
\bcbar[label=SDK 4]{7.03}
\bcskip{3pt}
\bcbar[label=SDK 10]{8.07}
\bcskip{3pt}
\bcbar[label=SDK 14]{10.53}
\bcskip{3pt}
\bcbar[label=SDK 7]{14.20}
\bcskip{3pt}
\bcbar[label=SDK 8]{18.39}
\bcskip{3pt}
\bcbar[label=SDK 9]{21.31}
\end{bchart}

\caption{Hodnoty najnižsej vyžadovanej verzie Android SDK}
\label{minSdkGraf}
\end{figure}

\begin{table}[htb]
\centering
  \begin{tabular}{|l l l r|}
    \hline
    \textbf{SDK} & \textbf{Android} & \textbf{Vydanie}& \textbf{\%} \\\hline\hline
    9 & Android 2.3 Gingerbread & November 2010 & 21,3 \\
    8 & Android 2.2.x Froyo & Jún 2010& 18,4 \\
    7 & Android 2.1.x Eclair & Január 2010 & 14,2 \\
    14 & Android 4.0 Ice Cream Sandwich & Október 2011 & 10,5 \\
    10 & Android 2.3.3 Gingerbread & Február 2011 & 8,1 \\
    4 & Android 1.6	Donut & September 2009 & 7,0 \\
    3 & Android 1.5	Cupcake & Máj 2009 & 5,6 \\
    15 & Android 4.0.3 Ice Cream Sandwich & December 2011 & 3,7 \\
    5 & Android 2.0 Eclair & November 2009 & 3,7 \\
    11 & Android 3.0.x Honeycomb & Február 2011 & 2,1\\
    \hline
  \end{tabular}
  \caption{Hodnoty najnižsej vyžadovanej verzie Android SDK}
  \label{tab:minSdk}
\end{table}

Až 25,64\,\% aplikácií je primárne určených na SDK verziu 19. SDK 19 bol vydaný koncom roka 2013 spolu s~verziou Android 4.4 \zv{KitKat}. Najčastejšie hodnoty primárnej verzie Android SDK obsahuje tabuľka \ref{tab:targetSdk} a graf na obrázku \ref{targetSdkGraf}.


\begin{figure}[htb]
\centering
\begin{bchart}[min=0,max=30,step=5,unit=\%]
\bcbar[label=SDK 8]{2.93}
\bcskip{3pt}
\bcbar[label=SDK 20]{3.78}
\bcskip{3pt}
\bcbar[label=SDK 18]{5.60}
\bcskip{3pt}
\bcbar[label=SDK 16]{5.68}
\bcskip{3pt}
\bcbar[label=SDK 22]{5.96}
\bcskip{3pt}
\bcbar[label=SDK 14]{6.28}
\bcskip{3pt}
\bcbar[label=SDK 15]{8.81}
\bcskip{3pt}
\bcbar[label=SDK 21]{11.70}
\bcskip{3pt}
\bcbar[label=SDK 17]{11.77}
\bcskip{3pt}
\bcbar[label=SDK 19]{25.64}
\end{bchart}

\caption{Hodnoty cieľovej verzie Android SDK}
\label{targetSdkGraf}
\end{figure}

\begin{table}[htb]
\centering
  \begin{tabular}{|l l l r|}
    \hline
    \textbf{SDK} & \textbf{Android} & \textbf{Vydanie}& \textbf{\%} \\\hline\hline
    19 & Android 4.4 KitKat & Október 2013 & 25,6 \\
    17 & Android 4.2\,--\,4.2.2 Jelly Bean & November 2012 & 11,8 \\
    21 & Android 5.0 Lollipop & November 2014 & 11,7 \\
    15 & Android 4.0.3 Ice Cream Sandwich & December 2011 & 6,8 \\
    14 & Android 4.0 Ice Cream Sandwich & Október 2011 & 6,3 \\
    22 & Android 5.1 Lollipop & Marec 2015 & 6,0\\
    16 & Android 4.1\,--\,4.1.1 Jelly Bean & Jún 2012 & 5,7 \\
    18 & Android 4.3 Jelly Bean & Júl 2013 & 5,6 \\
    20 & Android 4.4W Kitkat Watch & Jún 2014 & 3,8 \\
    8 & Android 2.2.x Froyo & Jún 2010& 2,9\\
    \hline
  \end{tabular}
  \caption{Hodnoty cieľovej verzie Android SDK}
  \label{tab:targetSdk}
\end{table}