\chapter{Databáza inštalačných APK súborov}
Základnou úlohou tejto práce je vytvoriť dostatočne veľkú databázu inštalačných APK balíčkov. Pre ďalšie potreby práce je požadované, aby veľká časť aplikácií pochádzala z~neoficiálnych zdrojov, čím sa zvyšuje pravdepodobnosť, že aplikácia bola modifikovaná alebo obsahuje malvér.

Naša databáza pozostáva približne z~20\,000 Android aplikácií. Tie boli zaobstarané v~časovom rozmedzí medzi novembrom 2015 a februárom 2016. Žiadna z~aplikácií nebola stiahnutá priamo z~obchodu \zv{Google Play}. Oficiálny zdroj \zv{Google Play Store} neumožňuje priame sťahovanie APK súborov. Žiadosť na stiahnutie APK súboru musí byť asociovaná s~Google účtom a konkrétnym Android zariadením prepojeným s~daným účtom pomocou jednoznačného identifikátoru \zv{ANDROID-ID}. Open source projekt \zv{Google Play Crawler}~\cite{gpCrawler} založený na neoficiálnom \zv{Google Play API}~\cite{gpApi} umožňuje limitovaný prístup, prehľadávanie a aj sťahovanie aplikácií z~\zv{Google Play Store}. Získavanie APK súborov pomocou \zv{Google Play Crawler} nebolo možné efektívne realizovať kvôli zastaralosti a funkčným obmedzeniam neoficiálneho \zv{Google Play API}, ktoré bolo naposledy upravené v~roku 2012.

Veľká časť APK súborov bola získaná s~využitím projektu \zv{Playdrone}. V~rámci tohto projektu bolo v~novembri 2014 z~\zv{Google Play} stiahnutých viac ako milión aplikácií dostupných pre zariadenie \zv{Galaxy Nexus} s~operátorom \zv{T-Mobile}~\cite{Viennot2014}. Naša databáza obsahuje 8200 najsťahovanejších aplikácií z~\zv{Google Play} v~období november 2014, ktoré boli stiahnuté z~archívu projektu \zv{Playdrone}. Ostatné aplikácie boli prevzaté z~neoficiálnych webových lokalít určených na distribúciu aplikácií, ale aj pomocou torrent súborov, alebo zo stránok na zdielanie ľubovoľného obsahu.

Celková veľkosť všetkých stiahnutých APK súborov je 192\,GB. Prehľad všetkých zdrojov APK súborov a ich počet zobrazuje tabuľka \ref{tab:stahovanie}. 

\begin{table}[htb]
\centering
  \begin{tabular}{|l r r|}
    \hline
    \textbf{Zdroj} & \textbf{Počet aplikácií} & \textbf{\%} \\\hline\hline
    Playdrone & 8200 & 40,9\\
    www.appsapk.com & 6470 & 32,3\\
    www.apkmaniafull.com & 2870 & 14,3\\
    www.androidapksfree.com & 1030 & 5,1\\
    www.zippyshare.com & 750 & 3,7\\
    torrenty & 550 & 2,7\\
    www.uloz.to & 190 & 0,9\\
    \midrule\hline
    Spolu & 20060 & \\
    \hline
  \end{tabular}
  \caption{Zdroje prevzatých APK súborov}
  \label{tab:stahovanie}
\end{table}


\section{Implementácia}
Viac ako 90\,\% aplikácií bolo stiahnutých automatizovane prostredníctvom aplikácie \zv{ApkDownloader} implementovanej v~rámci tejto práce. Aplikácia neposkytuje grafické užívateľské rozhranie, ale užívateľ môže zadávať parametre prostredníctvom príkazového riadku. \zv{ApkDownloader} funguje na jednoduchom princípe, keď najskôr získa zoznam URL odkazov na APK súbory, ktoré následne stiahne. Podporuje sťahovanie aplikácií získaných pomocou projektu \zv{Playdrone} alebo z~nasledujúcich neoficiálnych lokalít zameraných na distribúciu Android aplikácií: 
\begin{itemize}
 \item \zv{www.appsapk.com},
 \item \zv{www.apkmaniafull.com},
 \item \zv{www.androidapksfree.com}.
\end{itemize}

\noindent V~prípade sťahovania aplikácií z~archívu projektu \zv{Playdrone}~\cite{Viennot2014} je potrebné špecifikovať súbor s~metainformáciami obsahujúci odkazy na APK súbory. Tento súbor sa nachádza v~archíve projektu \zv{Playdrone}~\cite{archiveOrg}. Žiadny zo spomínaných zdrojov neposkytuje verejné API na prístup a sťahovanie súborov. Pri ostatných podporovaných zdrojoch \zv{ApkDownlaoder} vyhľadá odkazy na APK súbory v~HTML kóde príslušnej stránky. Vyhľadávanie odkazov je priamo závislé na HTML kóde danej webovej stránky a jeho zmena môže mať vplyv na korektné fungovanie aplikácie. Keďže je \zv{ApkDownloader} open source, môže byť jednoducho adaptovaný na zmenený HTML kód, ale aj rozšírený o~podporu sťahovania APK súborov z~nových lokalít. Za účelom rozšíriteľnosti a jednoduchej modifikovateľnosti poskytuje \zv{ApkDownlaoder} jednoduché verejné API. Použitie aplikácie \zv{ApkDownlaoder} je zobrazené v~ukážke kódu 4.1, ktorá reprezentuje telo metódy \zv{main} danej aplikácie. \\
\begin{lstlisting} [language=java, basicstyle=\small\color{black},frame=single,framesep=10pt, caption= {\zv{ApkDownloader} API a jeho použitie} \label{downlaoderCode}]
ApkLinkFinder finder = arguments.getLinkFinder();
finder.setMetadataFile(arguments.getMetadataFile());
finder.setNumberOfApks(arguments.getNumberOfApks());
List<String> urls = finder.findLinks();

ApkDownloader downloader = arguments.getApkDownloader();
downloader.setDownloadDirectory(arguments.getDownloadDirectory());
downloader.setNumberOfThreads(arguments.getNumberOfThreads());
downloader.setOverwriteExisting(arguments.isOverwriteExisting());
downloader.download(urls);
\end{lstlisting}
\mbox{}\\
\noindent Užívateľ pomocou parametrov špecifikuje z~ktorej podporovanej lokality chce APK súbory stiahnuť, ich želaný počet, umiestnenie prebraných súborov a maximálny počet súbežných preberaní. Pri vyhľadávaní URL odkazov je na prácu s~HTML súbormi použitá open source knižnica \zv{jsoup}. Pri sťahovaní sa využíva knižnica \zv{HtmlUnit}, ktorá poskytuje funkcionalitu internetového prehliadača. Na preberanie súborov z~URL odkazov je použitá knižnica \zv{Apache Commons IO}. 

Torrent súbory boli získané automatizovane s~využitím knižnice \zv{flux}~\cite{flux}. \zv{Flux} poskytuje API na vyhľadávanie a sťahovanie torrent súborov z~portálu \zv{www.torrentz.eu}. Tento nástroj je implementovaný v~prostredí NodeJS. Samotné APK súbory boli následne prevzaté prostredníctvom bežného torrent klienta.