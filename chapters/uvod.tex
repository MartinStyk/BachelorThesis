\chapter{Úvod}
Chytré mobilné zariadenia sú v~súčasnosti veľmi obľúbené a rozšírené. Na trhu s~mobilnými telefónmi dominujú smartfóny, ktoré neplnia len funkciu telefónu, ale je v~nich integrovaná aj funkcionalita fotoaparátu, multimediálneho prehrávača alebo GPS navigácie. Chytré zariadenia dnes obsahujú mnoho funkcií, na ktoré bol kedysi potrebný špecializovaný hardvér. Smartfóny a tablety obsahujú plnohodnotný operačný systém a poskytujú užívateľom možnosť výberu z~množstva aplikácií, ktoré môžu využívanie chytrého zariadenia zefektívniť alebo spríjemniť. Na trhu s~mobilnými operačnými systémami je dominantný systém Android. Užívatelia preferujú Android kvôli prívetivému užívateľskému rozhraniu a dobrej dostupnosti aplikácií. Medzi výrobcami je tento systém populárny vďaka voľne dostupnému zdrojovému kódu a môžnosti modifikácie a vyladenia pre potreby konkrétneho zariadenia.

V~tejto práci sa zaoberám získavaním a analýzou metainformácií o~inštalačných súboroch a aplikáciách pre mobilný operačný systém Android. Aplikácie pre Android sú distribuované formou inštalačných APK balíčkov. V~rámci práce je vytvorená databáza takýchto inštalačných súborov. 

Cieľom práce je analýza APK súborov za účelom získania metadát. Zámerom je získať metainformácie o~veľkom počte vlastností z~rôznych častí APK súboru. Metainformácie obsahujú údaje o~veľkosti aplikácií, počte a formáte multimediálnych súborov, certifikáte, komponentách a preferenciách aplikácie, ale aj o~súboroch obsiahnutých v~APK balíčku. Na základe týchto metadát sú na vzorke aplikácií z~vytvorenej databázy určené štatistické vlastnosti APK súborov. 

Hlavným dôvodom analýzy APK súborov je však bezpečnosť samotných aplikácií a s~tým súvisiaca bezpečnosť celého systému Android. Metadáta získané analýzou sú použité pri detekcii potenciálne škodlivých APK balíčkov.
V~súčasnosti existuje viacero metód detekcie škodlivých modifikovaných APK  súborov. V~tejto práci by som rád nadviazal na metódu založenú na podobnosti súborov obsiahnutých v~APK balíčkoch.
Primárnym cieľom pri detekcii modifikovaných APK súborov je využitie metadát na efektívne odhalenie aplikácií, ktoré boli pozmenené. Narozdiel od ostatných metód detekcie modifikovaných APK súborov je mojim cieľom poskytnúť užívateľovi detailný výstup obsahujúci rozdiely medzi dvojicou podobných APK balíčkov.

Výsledkom tejto práce je databáza APK súborov obsahujúca aplikácie pochádzajúce prevažne z~alternatívnych zdrojov. V~rámci práci je implementovaný nástroj na automatizované sťahovanie APK súborov. Ďalším výstupom je aplikácia \zv{ApkAnalyzer}, ktorá slúži na hromadnú analýzu APK súborov. Práca obsahuje štatistické dáta získané analýzou inštalačných súborov. V~práci je popísaný navrhnutý spôsob detekcie potenciálne škodlivých prebalených APK súborov. Navrhnutá metóda je prakticky implementovaná v~aplikácii \zv{ApkAnalyzer}.