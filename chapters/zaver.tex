\chapter{Záver}
Cieľom práce bolo vytvorenie rozsiahlej databázy inštalačných APK súborov pre operačný systém Android. Získané APK súbory mali byť analyzované za účelom získania metadát, nad ktorými mala byť vykonaná analýza štatistických vlastností APK súborov z~vytvorenej databázy. Ďalším cieľom bol návrh a implementácia jednoduchej metódy využívajúcej informácie o~APK súboroch na detekciu potenciálne škodlivých prebalených aplikácií. Všetky ciele vytýčené zadaním práce sa podarilo splniť.

V~rámci práce bola vytvorená databáza obsahujúca viac ako 20000 APK súborov. Získanie veľkého počtu APK súborov bolo možné vďaka automatizovaniu ich sťahovania prostredníctvom vyvinutého nástroja \zv{ApkDownloader}. Časť aplikácií bola získaná z~archívu vytvoreného v~práci \zv{Playdrone}, ktorý obsahuje aplikácie prevzaté z~oficiálneho obchodu \zv{Google Play}~\cite{Viennot2014}. Väčšina aplikácií však pochádza z~neoficiálnych zdrojov, čím sa zvyšuje pravdepodnosť prítomnosti škodlivého softvéru v~aplikáciách. 

S~využitím nástrojov reverzného inžinierstva boli APK súbory z~našej databázi analyzované. Práca obsahuje detailný popis štruktúry a obsahu APK balíčkov so zameraním na súbory a atribúty využívané pri ich analýze. Bol implementovaný program \zv{ApkAnalyzer}, ktorý poskytuje funkcionalitu na analýzu veľkého počtu APK súborov. Výstupom analýzy jedného APK balíčka je súbor obsahujúci metadáta o~danom balíčku.

Na základe informácií o~jednotlivých aplikáciách boli určené štatistické dáta nad množinou APK súborov. Vytvorená databáza poskytovala dostatočne veľký počet rôznych APK súborov na výpočet štatistických informácií.

Práca obsahuje návrh a implementáciu jednoduchej metódy detekcie prebalených APK balíčkov. Metóda je založená na zhode metadát a súborov obsiahnutých v~APK súboroch. Túto funkcionalitu poskytuje aplikácia \zv{ApkAnalyzer}, ktorá umožňuje detekovať dvojice nadmieru podobných APK súborov a zobraziť rozdiely medzi nimi. Pomocou navrhnutého algoritmu bolo detekovaných 161 dvojíc aplikácií, pri ktorých je veľká pravdepodbnosť modifikácie.

Programy \zv{ApkAnalyzer} a \zv{ApkDownloader} sú open source, môžu byť upravené a použité v~ďalších prácach zaoberajúcich sa danou problematikou. 